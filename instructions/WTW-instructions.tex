% Created 2016-08-31 Wed 10:11
\documentclass[DIV=15,halfparskip,11pt,headinclude]{scrartcl}
\usepackage[utf8]{inputenc}
\newcommand{\un}[1]{\ \textrm{#1}}
\newcommand{\rd}{\,\mathrm{d}}
\newcommand{\pd}[2]{\frac{\partial {#1}}{\partial {#2}}}
\newcommand{\Div}[1]{\nabla \cdot \mathbf{#1}}
\newcommand{\bm}[1]{\mathbf{#1}}
\usepackage[utf8]{inputenc}
\usepackage[T1]{fontenc}
\usepackage{fixltx2e}
\usepackage{graphicx}
\usepackage{longtable}
\usepackage{float}
\usepackage{wrapfig}
\usepackage{rotating}
\usepackage[normalem]{ulem}
\usepackage{amsmath}
\usepackage{textcomp}
\usepackage{marvosym}
\usepackage{wasysym}
\usepackage{amssymb}
\usepackage{hyperref}
\tolerance=1000
%\usepackage{minted}
\author{Mauro Werder}
\date{\today}
\title{ETH glaciology field-course 2018:\\ stream gauging}
\hypersetup{
  pdfkeywords={},
  pdfsubject={ETH glaciology field-course}
  }
\begin{document}

\section{How to log data with WTW conductivity sensor}

At measurement site:
\begin{enumerate}
\item if discharge is too low to fully submerge the sensor, make a
depression with the ice axe.
\item place sensor in stream (the sensor is fully waterproof),
  ideally with the sensor located in the central part of the stream
\item let the sensor temperature stabilise ($\sim$2min)
\item write down the background conductivity
\item set the datalogger:
\begin{itemize}
\item long press on ``STO''
\item make sure sampling interval is 1s
\item select long enough logging time, maybe 3min
\item hit ``continue'' to start the logging
\end{itemize}
\item signal to others to initiate injection
\item check the sensor readout during the tracer passage.  Make sure
  that:
\begin{itemize}
\item maximum concentration does not go $\sim$2x above maximal
  calibrated concentration (if not repeat with less salt)
\item that concentration rises by at least 50$\mathrm{\mu S/cm}$ to get
  a good signal to noise ratio (if not repeat with more salt)
\item the time interval is more than 15 seconds during which the
  signal is at least 10$\mathrm{\mu S/cm}$ above background (if not
  repeat but inject salt more slowly or further up in the stream.
  This may necessitate a larger quantity.)
\end{itemize}
\item check that the data has been logged: long-press ``RCL'' \& check
  that counter increased by 60 per minuted logged. (Alternatively go
  back with the arrow keys to check for current logging entries).
  \textbf{In several instances we mysteriously lost data of some
    traces, so please do this check after each trace.}
\end{enumerate}


Download:
\begin{itemize}
\item switch datalogger on
\item insert USB-stick
\item long press ``Menu/Enter''
\item got to ``Data storage''
\item got to ``Automatic data storage''
\item got to ``Output to USB''
\item once finished it will return to the previous screen, remove USB
  stick.
\end{itemize}
\end{document}
%%% Local Variables:
%%% mode: latex
%%% TeX-master: t
%%% End:
